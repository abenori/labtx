\documentclass[a4paper]{ltjsarticle}
\usepackage[hiragino-pron]{luatexja-preset}
\usepackage[T1]{fontenc}
\usepackage{lmodern}
\usepackage{url}
\usepackage{listings}
\usepackage[rgb,x11names]{xcolor}
\usepackage{xparse}
\usepackage{enumitem}
\lstdefinestyle{Lua}{%
  language=[5.2]Lua,
  basicstyle=\ttfamily,
  columns=spaceflexible,
  keywordstyle=\bfseries\color{Blue4},% language keywords
  keywordstyle=[2]\bfseries\color{RoyalBlue3},% std. library identifiers
  keywordstyle=[3]\bfseries\color{Purple3},% labels
  stringstyle=\bfseries\color{Coral4},% strings
  commentstyle=\gtfamily\color{Green4},% comments
  lineskip=-0.5\zw,
}
\lstset{style=Lua,tabsize=2,showspaces=false}
\DeclareRobustCommand{\labtx}{labtx}
\newcommand{\luafunc}[1]{\texttt{#1}}
%\NewDocumentCommand{\luafunc}{v}{\texttt{#1}}
\newcommand*{\luatable}[1]{\texttt{#1}}
\NewDocumentCommand{\luastring}{v}{``\texttt{#1}''}
%\NewDocumentCommand{\luavar}{v}{\texttt{#1}}
\newcommand*{\luavar}[1]{\texttt{#1}}
\NewDocumentCommand{\texcs}{v}{\texttt{#1}}
\renewcommand{\theenumi}{\inhibitglue(\arabic{enumi})\inhibitglue}
\renewcommand{\labelenumi}{\theenumi}
\title{\labtx}
\date{}
\begin{document}
\maketitle
Luaによる\BibTeX の実装です.

\section{使い方}
TeX LiveまたはW32TeXをインストールしておいてください.
拡張子がluaであるファイルを全てkpathseaが探せる場所においてください.
例えばTeX Liveの標準設定では\url{$TEXMF/scripts}以下に置くことができます.\if0$\fi
また,
\begin{itemize}
\item UNIX: labtx.luaへのリンクを適当なbinディレクトリに作る.
\item Windows (TeX Live): bin/win32/runscript.exeをbin/win32/labtx.exeとしてコピーする.
\item W32TeX: bin/win32/runscr.exeをbin/win32/labtx.exeとしてコピーする.
\end{itemize}
とします.
または,代わりに\url{labtx}(UNIX)または\url{labtx.bat}(Windows)をPATHの通っている場所におくことでも実行が可能になります.

\begin{lstlisting}[language=bash]
$ bibtex sample
\end{lstlisting}
としていた代わりに
\begin{lstlisting}[language=bash]
$ labtx sample
\end{lstlisting}
とします.つまり,sample.texを処理するには
\begin{lstlisting}[language=bash]
$ latex sample.tex
$ labtx sample
$ latex sample.tex
$ latex sample.tex
\end{lstlisting}
とします.
文字コードは(現在のところ)UTF-8に限定されています.



\section{データベースについて}
通常の.bibを読むことができます.
典型的には次のようになっています.
\begin{verbatim}
@article{reference,
   author = "Last, First",
   title = {Some title},
}
\end{verbatim}
これはreferenceという名前のついたarticleに関する情報です.
著者名とタイトルが定義されています.
本マニュアルでは,
\begin{itemize}
\item 各々のデータを「エントリー」
\item articleを「エントリータイプ」
\item referenceを「エントリーキー」
\item 著者名などの情報を「フィールド」
\item author = "Last, First"におけるauthorを「キー」"Last, First"を「値」
\end{itemize}
と呼ぶことにします.
エントリータイプ,エントリーキー,またフィールドのキーは大文字小文字を無視して処理されます.

また次のようなデータ
\begin{verbatim}
@string{str = "some string"}
@article{reference
   title = "Title and " # str
}
\end{verbatim}
に対しては,文字列の連結と置換が行われます.
たとえばこの例ではreference内のtitleに対する値が\luastring{Title and some string}と置き換えられます.
このような置き換えのルール(今の場合はstrを\luastring{some string}に置き換える)をマクロと呼ぶことにします.
マクロはこのようにデータベース内のstringエントリーを使っても定義できますし,スタイルファイル内で定義することもできます.


正確には次のようなEBNFで定義されたファイルを読みます.
\begin{verbatim}
Database = (Ignored '@' Entry)*
Ignored = [^@]*
Entry = Preamble|Comment|String|Data
Comment = "comment" ('{' Name '}'| '(' Name ')')
Preamble = "preamble" ('{' Name '}'| '(' Name ')')
String = "string" ('{' Fields '}'|'(' Fields ')')
Data = Entry_Type ('{' Entry_Key ',' Fields '}'|'(' Entry_Key ',' Fields ')')
Entry_Type = [^{(]*
Entry_Key = Name
Fields = Field? (,Field)* ','?
Field = Key '=' Value
Key = Name
Value = Name
Name = ([^{}"]+ | '"'[^"]*'"' | '{' Name '}')*
\end{verbatim}

\begin{itemize}
\item Ignoredは無視されます.通常空白と改行のみを含みます.(ただし,この部分をコメントとして用いることも可能です.)
\item Commentはコメントです.無視されます.
\item Preambleはそのままbblに書き出されることが想定されています.
\item Stringはマクロを定義します.これは後述の「文字列連結機能」にて用いられます.
\item Dataが文献情報を表します.
\end{itemize}


また文字列の連結および置換は次のように振る舞います.
\begin{verbatim}
Value = EachString ('#' EachString)*
EachString = ([^#{}"]* | '"' [^"]* '"' | '{' EachString '}')*
\end{verbatim}
各々のEachStringには以下の処理が施されます.
\begin{itemize}
\item 前後の空白は全て無視されます.
\item EachStringと同じ文字列がマクロとして定義されていた場合,対応する文字列に変更されます.
\item 最後に,最初及び最後の\verb|"{}|は削除されます.
\end{itemize}


\section{スタイルファイルの書き方}
Lua言語によりスタイルを記述することができます.kpathseaから見える場所に\url{labtx-<style>_bst.lua}として保存してください.(\url{<style>}はスタイル名.)
標準のplain, alpha, abbrv, unsrtに対応するファイルは既に用意されています.

スタイルファイルの中身は,Luaスクリプトファイルです.
グローバル変数BibTeXを通じ,各種設定などを行います.
多くの場合,次のような流れになるでしょう.
\begin{enumerate}
\item \luavar{BibTeX.blockseparator}に,テンプレート設定で使うセパレータを設定する.
\item \luavar{BibTeX.templates}と\luavar{BibTeX.formatters}に実際にthebibliography環境として出力する内容のテンプレートを設定する.
\item \luavar{BibTeX.crossref}にクロスリファレンスの設定を行う.
\item \luavar{BibTeX.sorting}にソートの設定をする.
\item \luavar{BibTeX.label}にラベル出力の設定をする.
\item \luafunc{BibTeX:outputthebibliography()}で出力を行う.
\end{enumerate}
順番に見ていきます.

\subsection{テンプレート設定}\label{subsec:テンプレート設定}
\luavar{BibTeX.templates},\luavar{BibTeX.formatters}および\luavar{BibTeX.blockseparator}を通じて設定を行います.
たとえば,エントリータイプarticleに対しては,著者,タイトル,ジャーナル,年をカンマ区切りで出し,最後にピリオドをつける場合は次のようにします.
\begin{lstlisting}
local Functions = require "labtx-funcs"

BibTeX.blockseparator = {{", ","."}}
BibTeX.templates["article"] = "[$<author>:<\\emph{|$<title>|}>:$<journal>:$<year>]"
function BibTeX.formatters:author(c)
	if c.fields["author"] == nil then return nil end
	local a = Functions.split_names(c.fields["author"])
	if #a <= 2 then
		return Functions.make_name_list(a,"{ff~}{vv~}{ll}{, jj}",{", "," and "},", et~al.")
	else
		return Functions.make_name_list(a,"{ff~}{vv~}{ll}{, jj}",{", ",", and "},", et~al.")
	end
end
\end{lstlisting}
\luavar{BibTeX.templates}に実際に出力される内容を設定します.
次のような書式で指定します.
\begin{itemize}
\item \luastring{[A:B:C...:X]}は「ブロック」を表します.各ブロックには「セパレータ」\luastring{<sep>}と「終端文字列」\luastring{<last>}が設定されており,\luastring{A<sep>B<sep>C...<sep>X<last>}というように出力されます.ただし,たとえば\luastring{B}が空文字列の場合は,\luastring{A<sep>C...<sep>X<last>}というように出力されます.
なお,このセパレータや終端文字列では,\luastring{.}が連続しないように処理がされます.ブロックはネストが可能です.
\item \luastring{$<A>}はフィールドAの出力を行います.Aがフィールドにない場合は空文字列になります.また\luastring{$<A|B|...|X>}と続けることもできて,この場合はA,B,...,Xの中で最初に定義されているものが出力されます.
\item \luastring{<A|B|C>}は,Bが空文字列ならば空文字列に,そうでないならば\luastring{ABC}という文字列になります.ネストが可能です.
\item 特殊文字は\luastring{%}でエスケープできます.
\end{itemize}

ブロックのセパレータと終端文字列は\luavar{BibTeX.blockseparator}で設定します.
中身は配列で,
\begin{lstlisting}
BibTeX.blockseparator = {
    {<ネストレベル1のセパレータ>,<ネストレベル1の終端文字列>},
    {<ネストレベル2のセパレータ>,<ネストレベル2の終端文字列>},
    ...
}
\end{lstlisting}
という形です.

\luastring{$<A|B|...|X>}で出力される各種フィールドの出力は\luavar{BibTeX.formatters}により整形されます.%$
その実体は関数で,キーnameのフィールドの整形を行う関数は
\begin{lstlisting}
function BibTeX.formatters:name(c)
-- 本体
end
\end{lstlisting}
という形で定義します.
戻り値は文字列です.
引数\luavar{c}には
\begin{itemize}
\item \luavar{c.key}にはエントリーキー
\item \luavar{c.type}にはエントリータイプ
\item \luavar{c.fields[name]}にはキーがnameのフィールドの中身
\end{itemize}
が入っています.
より詳しくは節\ref{sec:文献データ}を参照してください.
上のauthorの例ではモジュール\luavar{labtx-funcs}の提供する関数を使っています.
節\ref{sec:関数}を参照してください.

BibTeX.formattersの名前は実際のフィールド名である必要はありません.
たとえば
\begin{lstlisting}
BibTeX.templates["article"] = "$<author_editor>:$<title>"
function BibTeX.formatters:author_editor(c)
    if c.fields["author"] == nil then return c.fields["editor"]
    else return c.fields["author"]
end
\end{lstlisting}
とすると,\luastring{$<author_editor>}%$
は「authorが定義されていればauthorフィールドに,そうでなければeditorフィールド」という扱いになります.(つまり\luastring{$<author|editor>}と同等.)%$

少し発展的な内容です.
\begin{itemize}
\item ブロックの定義において,\luastring{[A:@S<sep>B:C]}とすると,Bの前のセパレータを\luastring{sep}に変更できます.
\item \luastring{$<A|(B)|C|...|X>}とすると,Bはフィールド名ではなく,テンプレートして解釈されます.たとえば,\luastring{$<author|(<edited by |\$<editor>|.)>}とすると,
\begin{itemize}
\item authorが定義されていればauthorフィールドそのまま.
\item authorが定義されていなく,editorが定義されていれば\luastring{edited by <editorフィールド>.}
\item authorもeditorも定義されていなければ空文字列
\end{itemize}
が出力されます.
\item formattersにもtemplatesのような書式が使えます.たとえば上の\luafunc{BibTeX.formatters:author\_editor}の例は
\begin{lstlisting}
BibTeX.formatters.author_editor = "$<author|editor>"
\end{lstlisting}
%$
と書くこともできます.
なお,ここでの\luastring{$<A>}%$
によるフィールド名の参照は,必ずフィールドの内容そのままとして解釈され,formattersによる整形は行われません.
\item formattersの関数の戻り値は原則文字列ですが,文字列の配列を返すこともできます.
これはブロックとして扱われます.
たとえば
\begin{lstlisting}
BibTeX.templates["article"] = "[$<author>:$<title_journal_year>]"
function BibTeX.formatters:title_journal_year(c)
    return {c.fields["title"],c.fields["journal"],c.fields["year"]}
end
\end{lstlisting}
と
\begin{lstlisting}
BibTeX.templates["article"] = "[$<author>:$<title>:$<journal>:$<year>]"
\end{lstlisting}
は等価です.
\end{itemize}

\subsection{クロスリファレンス}
クロスリファレンスの設定はBibTeX.corssrefに対して行います.
例としては次のようになります.
\begin{lstlisting}
BibTeX.crossref.templates["article"] = "[$<author>:$<title>:\\cite{$<crossref>}]"
\end{lstlisting}
%$
これにより,corssrefフィールドが定義されているarticleに対しては,その出力が上で指定されたものに変わります.
なお,\luavar{formatters}や\luavar{blockseparator}は\luavar{BibTeX.formatters}や\luavar{BibTeX.blockseparator}がそのまま使われます.
また,\luavar{BibTeX.crossref.templates["article"]}が定義されていない場合は\luavar{BibTeX.formatters["article"]}が使われます.

\subsubsection{クロスリファレンスの遺伝}
クロスリファレンスが行われると,親エントリーから子エントリーへとフィルードのコピーが行われます.
デフォルトでは,そのままのコピーが行われますが,この挙動は制御することができます.
たとえば
\begin{lstlisting}
BibTeX.crossref.inherit["article"]["book"] = {
    {"title","booktitle"},
    {{"author","editor"},"editor"},
    {{"A","B"},{"C","D"}}
}
\end{lstlisting}
とすると,親:article,子:bookというクロスリファレンスに対して
\begin{itemize}
\item titleはbooktitleにコピー
\item authorとeditorはeditorにコピー
\item A,BはC,Dの両方にコピー
\end{itemize}
が行われます.各々の項目に空文字列\luastring{}を指定すると,それは「全部」を表します.
たとえば
\begin{lstlisting}
BibTeX.crossref.inherit[""][""] = {
    {"title","booktitle"}
}
\end{lstlisting}
は全てのエントリータイプに対して,titleをbooktitleへとコピーします.
個別の指定は,\luastring{}による全てへの指定より優先されます.
たとえば
\begin{lstlisting}
BibTeX.crossref.inherit[""][""] = {
    {"title","booktitle"}
}
BibTeX.crossref.inherit["article"][""] = {
    {"title","subtitle"}
}
\end{lstlisting}
という指定は,articleからの場合に限りtitleをsubtitileに,それ以外はtitleをbooktitleにコピーします.

\subsubsection{その他の設定}
子エントリーに既にフィールドが存在している場合に上書きするかどうかは,\luavar{BibTeX.crossref.override}で制御します.
簡単な方法は
\begin{lstlisting}
BibTeX.crossref.override = true
\end{lstlisting}
とすることです.
これで全てのフィールドが上書きされます.(なお,デフォルトはfalseです.)
inheritと同様個別の定義を行うこともできます.
たとえば
\begin{lstlisting}
BibTeX.crossref.override["article"]["book"] = {
    {{"author","editor"},{"bookeditor"},true}
}
\end{lstlisting}
は親:articleのauthorかeditorフィールドが子:bookのbookeditorフィールドにコピーされる場合に上書きを許すことを意味します.
inheritと同様\luastring{}は全ての項目を表します.

その他以下の項目が設定できます.
\begin{itemize}
\item \luavar{BibTeX.crossref.mincrossrefs}:ここに設定されているだけのクロスリファレンスがあれば,エントリーが現在の参考文献一覧に追加されます.デフォルト2.
\item \luavar{BibTeX.crossref.reference\_key\_name}:クロスリファレンスを表すフィールドのキー名です.デフォルト\luastring{crossref}.
\end{itemize}

\subsection{ソート}
ソートに関する設定は,\luavar{BibTeX.sorting}で行います.
\begin{lstlisting}
BibTeX.sorting.targets = {"name","title","year"}
\end{lstlisting}
とすると,「名前」「タイトル」「年」の順番で比較されます.タイトルと年については,ほぼフィールド名そのまま\footnote{タイトルの頭文字のA, An, Theは取り除かれる.}で比較されます.名前については,デフォルトでは
\begin{itemize}
\item book, inbook: author/editor/key
\item proceedings: editor/organization/key
\item manual: author/organization/key
\item その他: author/key
\end{itemize}
のうち定義されている最初のものになります.
\luavar{BibTeX.sorting.targets}には上の\luastring{name}とフィールド名の他,\luastring{entry_key}(エントリーキー),\luastring{label}(ラベル)が指定できます.

実際に比較する値は,BibTeX.sorting.formattersで設定可能です.
\begin{lstlisting}
function BibTeX.sorting.formatters:name(c)
....
end
\end{lstlisting}
とすると,上のnameに対応する定義を上書きすることができます.

比較するための関数は,
\begin{itemize}
\item 一致しているか否かを返す\luafunc{BibTeX.sorting.equal}
\item $<$であるかを返す\luafunc{BibTeX.sorting.lessthan}
\end{itemize}
で設定できます.
デフォルトでは
\begin{lstlisting}
local function purify(s)
    return s:gsub("\\[a-zA-Z]*",""):gsub("[ -/:-@%[-`{-~]","")
end
function BibTeX.sorting.lessthan(a,b)
    return unicode.utf8.lower(purify(a)) < unicode.utf8.lower(purify(b))
end
function BibTeX.sorting.equal(a,b)
    return unicode.utf8.lower(purify(a)) == unicode.utf8.lower(purify(b))
end
\end{lstlisting}
と定義されています.

\subsection{ラベル}
thebibliography環境における
\begin{lstlisting}[language={[latex]TeX}]
\bibitem[label]{key} ....
\end{lstlisting}
のlabelの部分をラベルと呼ぶことにします.
デフォルトでは,著者などから自動的に生成されます.
ただし,shorthandフィールドがある場合には,その値が使われます.
ラベルの生成を押さえる(標準スタイルの「plain」に対応)には
\begin{lstlisting}
BibTeX.label.make = false
\end{lstlisting}
とします.

より細かく設定する場合は,\luavar{BibTeX.label.templates}と\luavar{BibTeX.label.formatters}を設定します.
設定の方法はテンプレート(項\ref{subsec:テンプレート設定})と同様です.
なお,同じラベル名が生成された場合,デフォルトでは末尾にa,b,c,...が追加されます.

なお,
\begin{lstlisting}
function BibTeX.label:make(c)
  ....
  return ...
end
\end{lstlisting}
と関数として定義すると,その戻り値そのものがラベルとして利用されます.

\subsection{出力}
最後に
\begin{lstlisting}
BibTeX:outputthebibliography()
\end{lstlisting}
とすることで,\url{.bbl}ファイルが出力されます.

\section{関数}\label{sec:関数}
有用そうな関数群やオリジナルの\BibTeX に存在していた関数が,モジュール\luavar{labtx-funcs}で定義されています.
\begin{lstlisting}
local Functions = require "labtx-funcs"
x = Functions.text_prefix(str,num)
\end{lstlisting}
のように使ってください.

\subsection{\luafunc{stable\_sort(list,comp)}}
配列\luavar{list}に対して,安定なソートを行います.
\luavar{comp}は比較関数です.
省略された場合は標準演算子 \luafunc{<} が使われます.

\subsection{\luafunc{text\_prefix(str,num)}}
\luavar{str}の先頭\luavar{num}バイトを返します.
ただし,文字を途中で切ることはなく,またコントロールシークエンス等や引数はバイト数に加算されません.
たとえば,
\begin{lstlisting}
text_prefix("aあい",2)
text_prefix("あいう",5)
\end{lstlisting}
はそれぞれ\luastring{aあ},\luastring{あい}を返します.\footnote{内部コードはUTF-8なので,\luastring{あ}や\luastring{い}は3byteです.この扱いはどうするか考え中…….}

\subsection{\luafunc{text\_length(str)}}
strのバイト数を返しますが,コントロールシークエンス等や引数は加算されません.

\subsection{\luafunc{string\_split(str,func)}}
検索関数\luavar{func}により\luavar{str}を分割して返します.
戻り値は二つの配列で,一つ目の配列には分割された文字列,二つ目の配列には分割文字列が入ります.
たとえば
\begin{lstlisting}
string_split("aXbYc",function(s) return s:find("[XY]") end)
\end{lstlisting}
は
\begin{lstlisting}
{"a","b","c"},{"X","Y"}
\end{lstlisting}
を返します.


\subsection{\luafunc{change\_case(str,format)}}
大文字小文字の変換を行います.ただし,中括弧の中は処理されません.
\luavar{format}は\luastring{t},\luastring{u},\luastring{l}のどれかで,
\begin{itemize}
\item \luastring{u},\luastring{l}はそれぞれ大文字,小文字への変換を表す.
\item \luastring{t}は小文字への変換を行うが,一文字目及び\luastring{: *}で表される文字の次の文字は変換されない.
\end{itemize}


\subsection{\luafunc{split\_names(names[,seps])}}
複数名の名前からなる文字列から,各人の名前の入った配列を得ます.
人と人との区切りを配列\luavar{seps}で与えることができます.(配列中のいずれかにマッチした部分で区切られる.)
\luavar{seps}のデフォルトは\verb|{"[aA][nN][dD]"}|です.

\subsection{\luafunc{get\_name\_parts(names)}}
名前からfirst name,last name,von part,jr partの四つの部分を抽出します.
戻り値は
\begin{lstlisting}
{first = <first part>, last = <last part>, von = <von part>, jr = <jr part>}
\end{lstlisting}
で,各々の部分は
\begin{lstlisting}
{parts = <array of name>, seps = <separator of names>}
\end{lstlisting}
です.%\verb|<array of name>|は各部分の名前が配列で,\verb|<separator of names>|はnames内で使われていた区切り記号を表します.
例えばvon-von Last Last, First, jrに対しては,次のように返ります.
\begin{lstlisting}
{
	first = {parts = {"First"}, seps = {}},
	last = {parts = {"Last","Last"}, seps = {" "}},
	von = {parts = {"von", "von"}, seps = {"-"}},
	jr = {pars = {"jr"}, seps = {}}
}
\end{lstlisting}

この関数は,次のルールに基づき名前を分解します.
\begin{enumerate}
\item \luastring{[ ,~\t%-]+}に該当するパターン\footnote{Luaの意味でのパターン}で区切り,配列を生成する.
\item 1で区切られた際に用いられた区切り文字のうち,最初の一文字がカンマ「,」のものの数を数える.この数に基づき,次の三つのパターンのどれかと見なす.
\begin{enumerate}
\item カンマがない:First von Lastのパターン.頭から見てvonと見なされるパターンの前までがFirst,後ろから見てvonと見なされるパターンの後ろまでがLast.vonがない場合は1で区切られたうちの最後の一つのみがLast.(ただし,区切り文字が\verb|"-"|のものはまとめて考える.例えば\luastring{First Last Last}のLastは\luastring{Last}であるが,\luastring{First Last-Last}ならば\luastring{Last-Last}である.)
\item カンマが一つ:von Last, Firstのパターン.von LastからLastを抜き出す処理は(a)と同じ.
\item カンマが二つ:von Last, Jr, Firstのパターン.von LastからLastを抜き出す処理は(a)と同じ.
\end{enumerate}
\item 2における「vonと見なされるパターン」とは,(基本的には)\footnote{実際には中括弧内や,コントロールシークエンスで定義されたアクセントなども考慮に入れる.}最初に現れたアルファベットが小文字であるもののことである.
\end{enumerate}


\subsection{\luafunc{forat\_name\_by\_parts(nameparts,format)}}
\luavar{format}にて指定された書式に基づき,名前の整形を行います.
\luavar{nameparts}は\luafunc{get\_name\_parts}で得られる戻り値と同じかたちで与えます.
\luavar{format}で与える書式は次の形です.
\begin{quote}
\begin{verbatim}
<str1>{<before1><name1><after1>}<str2>{<before2><name2><after2>}...
\end{verbatim}
\end{quote}
\begin{itemize}
\item \verb|<str1>|はそのまま出力される.
\item \verb|<name1>|は\luastring{l},\luastring{ll},\luastring{f},\luastring{ff},\luastring{v},\luastring{vv},\luastring{j},\luastring{jj}の何れか.Last name,First name,von part,jr partに対応し,二つ続いているものは名前全体を,そうでないものは短縮形を出力する.
\item \verb|<before1>|はそのまま出力される.ただし\verb|<name1>|に対応する部分がない場合,出力されない.
\item \verb|<after1>|は\verb|{<sep1>}<after1_>|か\verb|<after1_>|(中括弧なし)の何れかである.\verb|<sep1>|は\verb|<name1>|の各部分をつなぐ文字として使われ,\verb|<after1_>|は次の部分とのつなぐ文字として使われる.\verb|<sep1>|が省略された場合や,\luastring{~}であった場合は,空白(\luastring{ }か\luastring{~})が状況に応じて使われる.もし常に\luastring{~}を出力したい場合は,\luastring{~~}を指定する.
\item \verb|<str2>|等も同様.
\end{itemize}

\subsection{\luafunc{format\_name(name,format)}}
\BibTeX のformat.names\$と似た関数です.
中身は
\begin{lstlisting}
return forat_name_by_parts(get_name_parts(name),format)
\end{lstlisting}
です.

\subsection{\luafunc{make\_name\_list(namearray, format, separray[, etalstr])}}
複数人の名前の配列から文字列を作ります.
\luavar{separray}の長さを\luavar{k},\luavar{namearray}の長さを\luavar{n}とすると,
\begin{quote}
\begin{verbatim}
namearray[1]separray[1]namaearray[2]separray[2] .... 
namearray[n - k + 1]separray[2] ...
namearray[n - 1]separray[k]namearray[n]
\end{verbatim}
\end{quote}
という文字列を生成します.(実際には改行無し.)
ただし,\luavar{namearray}の各項は\luavar{format}に従い整形され(書式は\luafunc{format\_name\_by\_parts}と同様),またもし\luavar{namearray[n]}が\luastring{others}の場合は,\luavar{namearray[n]}は\luavar{etalstr}に置き換えられます.
デフォルトでは\luavar{etalstr}は空文字列です.

\subsection{\luafunc{remove\_TeX\_cs(s)}}
\luavar{s}から\TeX のコントロールシークエンスを取り除いた文字列を得ます.

\section{文献データ}\label{sec:文献データ}
文献データは以下のようなテーブルに格納されています.
変数名を\luavar{Citation}とします.
\begin{description}[style=nextline]
\item[\luavar{Citation.type}]
エントリータイプ
\item[\luavar{Citation.key}]
エントリーキー
\item[\luavar{Citation.fields}]
フィールドが格納されているテーブル.マクロなどが施された結果が帰る.
\item[function \luafunc{Citation:clone()}]
自分の複製を作ります.
\item[function \luafunc{Citation:set\_field(key,cite,key1)}]
文献データ\luavar{cite}のキー\luastring{key1}のフィールドを\luastring{key}に設定します.
\item[function \luafunc{Citation:get\_raw\_field(key)}]
キー\luastring{key}のフィールドの生の値(マクロなど適用前)を返します.
\end{description}


\section{変数\luavar{BibTeX}}
変数\luavar{BibTeX}には,現在の\labtx の状態が格納されています.
\subsection{各種状態}
\begin{description}[style=nextline]
\item[\luavar{BibTeX.style}]
スタイル名.
\item[\luavar{BibTeX.cites}]
引用されている文献一覧からなる配列.各々の中身は節\ref{sec:文献データ}の通り.
\item[\luavar{BibTeX.db}]
読み込まれたデータベースを表すテーブル.エントリーキー\luastring{key}には
\begin{lstlisting}
BibTeX.db["key"]
\end{lstlisting}
でアクセスできる.各々の中身は節\ref{sec:文献データ}の通り.
\item[\luavar{BibTeX.aux}]
\url{aux}ファイル名.
\item[\luavar{BibTeX.aux\_contents}]
\url{aux}ファイル名の中身.\url{aux}の各行の
\begin{verbatim}
\somecs{arg1}[arg2](arg3)
\end{verbatim}
という行から
\begin{lstlisting}
{somecs = {
	{arg = "arg1", open = "{", close = "}"},
	{arg = "arg2", open = "[", close = "]"},
	{arg = "arg3", open = "(", close = ")"}
}}
\end{lstlisting}
というテーブルが生成されて,ここに格納されている.
括弧は上記の\luastring{{}}, \luastring{[]}, \luastring{()}が認識され,対応がとれているものとして扱われる.
\end{description}

\subsection{関数}
\begin{description}[style=nextline]
\item[\luafunc{BibTeX:output(str)}]
\url{bbl}への出力を行う.
\item[\luafunc{BibTeX:outputline(str)}]
\url{bbl}への一行出力を行う.
\item[\luafunc{BibTeX:outputthebibliography()}]
項\ref{subsec:出力設定}に従い\url{bbl}へのthebibliography環境の出力を行う.
\item[\luafunc{BibTeX:warning(str)}]
文字列\luavar{str}を警告として出力する.出力は標準出力および\url{blg}に対して行われる.
\item[\luafunc{BibTeX:error(str,exit\_code)}]
文字列\luavar{str}をエラーとして出力し,終了コード\luavar{exit\_code}でプログラムを終了する.
出力は標準エラー出力および\url{blg}に対して行われる.
\item[\luafunc{BibTeX:log(str)}]
\url{blg}に\luavar{str}を出力する.
\item[\luafunc{BibTeX:message(str)}]
標準出力に\luavar{str}を出力する.

\end{description}

\subsection{出力設定}\label{subsec:出力設定}
\begin{description}[style=nextline]
\item[\luavar{BibTeX.templates}]
出力される\texcs{\bibitem}のフォーマットを指定する.
書式は項\ref{subsec:テンプレート設定}に基づく.
\item[\luavar{BibTeX.formatters}]
\luavar{BibTeX.templates},\luavar{BibTeX.crossref.templates}で使われる整形用の関数.
書式は項\ref{subsec:テンプレート設定}に基づく.
\item[\luavar{BibTeX.blockseparator}]
\luavar{BibTeX.templates},\luavar{BibTeX.crossref.templates}におけるブロックの区切り文字.
\item[\luavar{BibTeX.crossref}]
クロスリファレンスの遺伝を設定する.
\item[\luavar{BibTeX.crossref.templates}]
クロスリファレンスが定義されている場合に使われるフォーマット.
値が定義されていない場合,\luavar{BibTeX.templates}が使われる.
\item[\luavar{BibTeX.sorting.targets}]
ソートの際に使われるフィールドキーを並べた配列.(正確には,テーブル\luavar{BibTeX.sorting.formatters}のキーを指定する.つまり,\luastring{key}を指定すると,関数\luafunc{BibTeX.sorting.formatters:key}が呼び出された結果が使われる.)
\item[\luavar{BibTeX.sorting.lessthan}, \luavar{BibTeX.sorting.equal}]
ソートのための比較関数.
\item[\luavar{BibTeX.sorting.formatters}]
ソート時のフィールドの整形関数からなるテーブル.比較の際に\luavar{self}と文献情報(節\ref{sec:文献データ})が渡されて実行される.
\item[\luafunc{BibTeX.sorting.label:make}]
\luavar{self}と文献情報(節\ref{sec:文献データ})を受け取り,ラベル名を返す関数を設定する.
ラベルを作らない場合はnilを設定する.
\item[\luafunc{self.label:add\_suffix}]
同一のラベル名があった場合に,ラベル名を変更する処理をおこなう関数を設定する.デフォルトでは末尾にa,b,c,...を付加する.
文献情報からなる配列(ソート済み)を受け,やはり配列を返す.
\item[\luafunc{self.label:modify\_citations}]
出力直前に実行される関数.最後の段階で文献情報を調整することができる.
文献情報からなる配列(ソート済み)を受け,やはり配列を返す.
\end{description}

\section{デバッグ}
次のようにしておくと,デバッグに有用な情報がはき出されたりする……かもしれません.
\begin{lstlisting}
local labtxdebug = require "labtx-debug"
labtxdebug.debugmode = true -- デバッグモードON

-- 以下スタイルファイルの処理
\end{lstlisting}


\end{document}


