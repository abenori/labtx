\documentclass[a4paper,lualatex]{jlreq}
\usepackage[hiragino-pron]{luatexja-preset}
\usepackage[T1]{fontenc}
\usepackage{lmodern}
\usepackage{url}
\usepackage{listings}
\usepackage[rgb,x11names]{xcolor}
\usepackage{xparse}
\usepackage{enumitem}
\usepackage{jslogo}
\lstdefinestyle{Lua}{%
  language=[5.3]Lua,
  basicstyle=\ttfamily,
  columns=spaceflexible,
  keywordstyle=\bfseries\color{Blue4},% language keywords
  keywordstyle=[2]\bfseries\color{RoyalBlue3},% std. library identifiers
  keywordstyle=[3]\bfseries\color{Purple3},% labels
  stringstyle=\bfseries\color{Coral4},% strings
  commentstyle=\gtfamily\color{Green4},% comments
  lineskip=-0.5\zw,
}
\lstset{style=Lua,tabsize=2,showspaces=false}
\DeclareRobustCommand{\labtx}{labtx}


\newcommand{\luafunc}[1]{\texttt{#1}}
%\NewDocumentCommand{\luafunc}{v}{\texttt{#1}}
\newcommand*{\luatable}[1]{\texttt{#1}}
\makeatletter
\def\luastring{\@ifstar{\luastring@s}{\luastring@}}
\NewDocumentCommand{\luastring@}{v}{``\texttt{#1}''}
\newcommand*{\luastring@s}[1]{``\texttt{#1}''}
\makeatother
%\NewDocumentCommand{\luavar}{v}{\texttt{#1}}
\newcommand*{\luavar}[1]{\texttt{#1}}
\NewDocumentCommand{\texcs}{v}{\texttt{#1}}
\renewcommand{\theenumi}{(\arabic{enumi})}
\renewcommand{\labelenumi}{\theenumi}
\title{\labtx}
\date{}
\jlreqsetup{itemization_labelsep={enumerate=0.5\zw}}
\begin{document}
\maketitle
Luaによる\BibTeX の実装です.

\section{インストール}
TeX LiveまたはW32TeXとともに使うことを想定しています.
拡張子がluaであるファイルを全てkpathseaが探せる場所においてください.
例えば\url{$TEXMF/srcpts/labtx}\if0$\fi 以下にすべておきます.
その後
\begin{itemize}
\item UNIX: labtx.luaへのリンクを適当なbinディレクトリに作る.
\item Windows (TeX Live): bin/win32/runscript.exeをbin/win32/labtx.exeとしてコピーする.
\item W32TeX: bin/win32/runscr.exeをbin/win32/labtx.exeとしてコピーする.
\end{itemize}
とします.

\section{使い方}
\begin{lstlisting}[language=bash]
$ bibtex sample
\end{lstlisting}
としていた代わりに
\begin{lstlisting}[language=bash]
$ labtx sample
\end{lstlisting}
とします.つまり,sample.texを処理するには
\begin{lstlisting}[language=bash]
$ latex sample.tex
$ labtx sample
$ latex sample.tex
$ latex sample.tex
\end{lstlisting}
とします.
文字コードは(現在のところ)UTF-8に限定されています.



\section{データベースについて}
スタイルファイルの作り方の前に,言葉の定義を兼ねて読み込むデータベースについて書いておきます.
データベースは典型的には次のようになっています.
\begin{verbatim}
@article{reference,
   author = "Last, First",
   title = {Some title},
}
\end{verbatim}
\begin{itemize}
\item 各々のデータを「エントリー」
\item articleを「エントリータイプ」
\item referenceを「エントリーキー」
\item 著者名などの情報を「フィールド」
\item author = "Last, First"におけるauthorを「キー」"Last, First"を「値」
\end{itemize}
と呼ぶことにします.
エントリータイプ,エントリーキー,またフィールドのキーは大文字小文字を無視して処理されます.

また次のようなデータ
\begin{verbatim}
@string{str = "some string"}
@article{reference
   title = "Title and " # str
}
\end{verbatim}
に対しては,文字列の連結と置換が行われます.
たとえばこの例ではreference内のtitleに対する値が\luastring{Title and some string}と置き換えられます.
このような置き換えのルール(今の場合はstrを\luastring{some string}に置き換える)をマクロと呼ぶことにします.
%
%
%正確には次のようなEBNFで定義されたファイルを読みます.
%\begin{verbatim}
%Database = (Ignored '@' Entry)*
%Ignored = [^@]*
%Entry = Preamble|Comment|String|Data
%Comment = "comment" ('{' Name '}'| '(' Name ')')
%Preamble = "preamble" ('{' Name '}'| '(' Name ')')
%String = "string" ('{' Fields '}'|'(' Fields ')')
%Data = Entry_Type ('{' Entry_Key ',' Fields '}'|'(' Entry_Key ',' Fields ')')
%Entry_Type = [^{(]*
%Entry_Key = Name
%Fields = Field? (,Field)* ','?
%Field = Key '=' Value
%Key = Name
%Value = Name
%Name = ([^{}"]+ | '"'[^"]*'"' | '{' Name '}')*
%\end{verbatim}
%
%\begin{itemize}
%\item Ignoredは無視されます.通常空白と改行のみを含みます.(ただし,この部分をコメントとして用いることも可能です.)
%\item Commentはコメントです.無視されます.
%\item Preambleはそのままbblに書き出されることが想定されています.
%\item Stringはマクロを定義します.これは後述の「文字列連結機能」にて用いられます.
%\item Dataが文献情報を表します.
%\end{itemize}
%
%
%また文字列の連結および置換は次のように振る舞います.
%\begin{verbatim}
%Value = EachString ('#' EachString)*
%EachString = ([^#{}"]* | '"' [^"]* '"' | '{' EachString '}')*
%\end{verbatim}
%各々のEachStringには以下の処理が施されます.
%\begin{itemize}
%\item 前後の空白は全て無視されます.
%\item EachStringと同じ文字列がマクロとして定義されていた場合,対応する文字列に変更されます.
%\item 最後に,最初及び最後の\verb|"{}|は削除されます.
%\end{itemize}

\section{スタイルファイルの作り方}
スタイルファイル名\url{<style>}の実体は,\url{labtx-<style>_bst.lua}というLuaスクリプトファイルです.
次の形をとります.
\begin{lstlisting}
local style = BibTeX:get_default_style() -- スタイルのひな形の取得
-- <styleに設定を入れていく>
return style -- 設定したスタイルを返す
\end{lstlisting}

実際に\luavar{style}に設定を入れていく部分は次の流れになります.
\begin{enumerate}
\item \luavar{style.templates}と\luavar{style.formatters}にエントリータイプごとの出力テンプレートを設定する.
\item \luavar{style.crossref}にクロスリファレンスの設定を行う.
\item \luavar{style.sorting}にソートの設定をする.
\item \luavar{style.label}にラベル出力の設定をする.
\end{enumerate}


順番に見ていきます.

\subsection{テンプレート設定}\label{subsec:テンプレート設定}
\luavar{style.templates}と\luavar{style.formatters}を通じて設定を行います.
以下の例を見てみます.
\begin{lstlisting}
local Functions = require "labtx-funcs" -- 便利関数ロード

style.templates["article"]
  = "[$<author>:<\\emph{|$<title>|}>:$<journal>:$<year>]"
function style.formatters:title(c)
  if c.fields["title"] == nil then return nil end
  return c.fields["title"]:upper() -- 大文字にする
end
function style.formatters:author(c)
	if c.fields["author"] == nil then return nil end
	local a = Functions.split_names(c.fields["author"])
	if #a <= 2 then
		return Functions.make_name_list(a,
		  "{ff~}{vv~}{ll}{, jj}",{", "," and "},", et~al.")
	else
		return Functions.make_name_list(a,
		  "{ff~}{vv~}{ll}{, jj}",{", ",", and "},", et~al.")
	end
end
\end{lstlisting}
lbtx-funcsについては\ref{sec:関数}節をご覧ください.
\luavar{style.templates["article"]}にエントリータイプarticleの出力設定を入れます.
上の例の通り,特殊な書式を持った文字列で指定します.
\begin{itemize}
\item \luastring{[A:B:C...:X]}は「ブロック」を表します.各ブロックには「セパレータ」\luastring{<sep>}と「終端文字列」\luastring{<last>}が設定されており,\luastring{A<sep>B<sep>C...<sep>X<last>}というように出力されます.ただし,たとえば\luastring{B}が空文字列の場合は,\luastring{A<sep>C...<sep>X<last>}というように出力されます.
なお,このセパレータや終端文字列では,\luastring{.}が連続しないように処理がされます.ブロックはネストが可能です.
セパレータと終端文字列は\luavar{style.blockseparator}で指定できます(後述).デフォルトはセパレータが\luastring{,},終端文字列が\luastring{.}です.
\item \luastring{$<A>}はフィールドAの出力を行います.Aがフィールドにない場合は空文字列になります.また\luastring{$<A|B|...|X>}と続けることもできて,この場合はA,B,...,Xの中で最初に定義されているものが出力されます.
\item \luastring{<A|B|C>}は,Bが空文字列ならば空文字列に,そうでないならば\luastring{ABC}という文字列になります.ネストが可能です.
\item 特殊文字は\luastring{%}でエスケープできます.
\end{itemize}

ブロックのセパレータと終端文字列は\luavar{style.blockseparator}で設定します.
中身は配列で,
\begin{lstlisting}
style.blockseparator = {
    {<ネストレベル1のセパレータ>,<ネストレベル1の終端文字列>},
    {<ネストレベル2のセパレータ>,<ネストレベル2の終端文字列>},
    ...
}
\end{lstlisting}
という形です.

\luastring{$<A>}で出力されるフィールドAの出力を,\luavar{style.formatters}により整形することができます.
その実体は関数で,\luastring{$<A>}の整形を行う関数は
\begin{lstlisting}
function style.formatters:A(c)
-- 本体
end
\end{lstlisting}
という形で定義します.
戻り値は文字列です.
引数\luavar{c}はテーブルで,
\begin{itemize}
\item \luavar{c.key}にはエントリーキー
\item \luavar{c.type}にはエントリータイプ
\item \luavar{c.fields[name]}にはキーがnameのフィールドの中身
\end{itemize}
が入っています.
より詳しくは節\ref{sec:文献データ}を参照してください.
上のauthorの例ではモジュール\luavar{labtx-funcs}の提供する関数を使っています.
節\ref{sec:関数}を参照してください.

style.formattersの名前は実際のフィールド名である必要はありません.
たとえば
\begin{lstlisting}
style.templates["article"] = "$<author_editor>:$<title>"
function style.formatters:author_editor(c)
    if c.fields["author"] == nil then return c.fields["editor"]
    else return c.fields["author"]
end
\end{lstlisting}
とすると,\luastring{$<author_editor>}は「authorが定義されていればauthorフィールドに,そうでなければeditorフィールド」という扱いになります.(つまり\luastring{$<author|editor>}と同等.)%$

少し発展的な内容です.
\begin{itemize}
\item ブロックの定義において,\luastring{[A:@S<sep>B:C]}とすると,Bの前のセパレータを\luastring{sep}に変更できます.
\item \luastring{$<A|(B)|C|...|X>}とすると,Bはフィールド名ではなく,テンプレートして解釈されます.たとえば,\luastring{$<author|(<edited by |\$<editor>|.)>}とすると,
\begin{itemize}
\item authorが定義されていればauthorフィールドそのまま.
\item authorが定義されていなく,editorが定義されていれば\luastring{edited by <editorフィールド>.}
\item authorもeditorも定義されていなければ空文字列
\end{itemize}
が出力されます.
\item formattersにもtemplatesのような書式が使えます.たとえば上の\luavar{style.formatters:author\_editor}の例は
\begin{lstlisting}
style.formatters.author_editor = "$<author|editor>"
\end{lstlisting}
%$
と書くこともできます.
なお,ここでの\luastring{$<A>}によるフィールド名の参照は,必ずフィールドの内容そのままとして解釈され,formattersによる整形は行われません.
\item formattersの関数の戻り値は原則文字列ですが,文字列の配列を返すこともできます.
これはブロックとして扱われます.
たとえば
\begin{lstlisting}
style.templates["article"] = "[$<author>:$<title_journal_year>]"
function style.formatters:title_journal_year(c)
    return {c.fields["title"],c.fields["journal"],c.fields["year"]}
end
\end{lstlisting}
と
\begin{lstlisting}
style.templates["article"] = "[$<author>:$<title>:$<journal>:$<year>]"
\end{lstlisting}
は等価です.
\end{itemize}

\subsection{クロスリファレンス}
クロスリファレンスの設定はstyle.corssrefに対して行います.
例としては次のようになります.
\begin{lstlisting}
style.crossref.templates["article"] 
  = "[$<author>:$<title>:\\cite{$<crossref>}]"
\end{lstlisting}
\if0$\fi
これにより,corssrefフィールドが定義されているarticleに対しては,その出力が上で指定されたものに変わります.
なお,\luavar{style.formatters}や\luavar{style.blockseparator}はそのまま使われます.
また,\luavar{style.crossref.templates["article"]}が定義されていない場合は\luavar{style.formatters["article"]}が使われます.

\subsubsection{クロスリファレンスの遺伝}
クロスリファレンスが行われると,親エントリーから子エントリーへとフィルードのコピーが行われます.
デフォルトでは,そのままのコピーが行われますが,この挙動は制御することができます.
たとえば
\begin{lstlisting}
style.crossref.inherit["article"]["book"] = {
    {"title","booktitle"},
    {{"author","editor"},"editor"},
    {{"A","B"},{"C","D"}}
}
\end{lstlisting}
とすると,親:article,子:bookというクロスリファレンスに対して
\begin{itemize}
\item titleはbooktitleにコピー
\item authorとeditorはeditorにコピー
\item A,BはC,Dの両方にコピー
\end{itemize}
が行われます.各々の項目に空文字列\luastring{}を指定すると,それは「全部」を表します.
たとえば
\begin{lstlisting}
style.crossref.inherit[""][""] = {
    {"title","booktitle"}
}
\end{lstlisting}
は全てのエントリータイプに対して,titleをbooktitleへとコピーします.
個別の指定は,\luastring{}による全てへの指定より優先されます.
たとえば
\begin{lstlisting}
style.crossref.inherit[""][""] = {
    {"title","booktitle"}
}
style.crossref.inherit["article"][""] = {
    {"title","subtitle"}
}
\end{lstlisting}
という指定は,articleからの場合に限りtitleをsubtitileに,それ以外はtitleをbooktitleにコピーします.

\subsubsection{その他の設定}
子エントリーに既にフィールドが存在している場合に上書きするかどうかは,\luavar{style.crossref.override}で制御します.
簡単な方法は
\begin{lstlisting}
style.crossref.override = true
\end{lstlisting}
とすることです.
これで全てのフィールドが上書きされます.(なお,デフォルトはfalseです.)
inheritと同様個別の定義を行うこともできます.
たとえば
\begin{lstlisting}
style.crossref.override["article"]["book"] = {
    {{"author","editor"},{"bookeditor"},true}
}
\end{lstlisting}
は親:articleのauthorかeditorフィールドが子:bookのbookeditorフィールドにコピーされる場合に上書きを許すことを意味します.
inheritと同様\luastring{}は全ての項目を表します.

その他以下の項目が設定できます.
\begin{itemize}
\item \luavar{style.crossref.mincrossrefs}:ここに設定されているだけのクロスリファレンスがあれば,エントリーが現在の参考文献一覧に追加されます.デフォルト2.
\item \luavar{style.crossref.reference\_key\_name}:クロスリファレンスを表すフィールドのキー名です.デフォルト\luastring{crossref}.
\end{itemize}

\subsection{ソート}
ソートに関する設定は,\luavar{style.sorting}で行います.
\begin{lstlisting}
style.sorting.targets = {"name","title","year"}
\end{lstlisting}
とすると,「名前」「タイトル」「年」の順番で比較されます.
実際に比較する値は,style.sorting.formattersで設定可能です.
\begin{lstlisting}
function style.sorting.formatters:name(c)
....
end
\end{lstlisting}
とすると,上のnameに対応する定義を上書きすることができます.

比較するための関数は,
\begin{itemize}
\item 一致しているか否かを返す\luafunc{style.sorting.equal}
\item $<$であるかを返す\luafunc{style.sorting.lessthan}
\end{itemize}
で設定できます.
どちらも二つの文字列をとり,bool値を返す関数です.


\subsection{ラベル}
thebibliography環境における
\begin{lstlisting}[language={[latex]TeX}]
\bibitem[label]{key} ....
\end{lstlisting}
のlabelの部分をラベルと呼ぶことにします.
デフォルトでは,著者などから自動的に生成されます.
ただし,shorthandフィールドがある場合には,その値が使われます.
ラベルの生成を押さえる(標準スタイルの「plain」に対応)には
\begin{lstlisting}
style.label.make = false
\end{lstlisting}
とします.

より細かく設定する場合は,\luavar{style.label.templates}と\luavar{style.label.formatters}を設定します.
設定の方法はテンプレート(項\ref{subsec:テンプレート設定})と同様です.
なお,同じラベル名が生成された場合,デフォルトでは末尾にa,b,c,...が追加されます.
末尾につく文字列を変更するには,\luavar{style.label.suffix}を再定義します.
\begin{lstlisting}
-- 同じラベルを持つもののうちn番目のものの末尾文字列を返す
function style.label:suffix(n)
  -- 1,2,3とつける
  return string.format("%s",n)
end
\end{lstlisting}

なお,
\begin{lstlisting}
function style.label:make(c)
  ....
  return ...
end
\end{lstlisting}
と関数として定義すると,その戻り値そのものがラベルとして利用されます.


\section{言語機能}
言語ごとに出力を切り替えるための簡単な機構が用意されています.
\subsection{言語ごとの設定}
\luavar{style.languages.<言語>}以下に,これまでと同様の設定を入れます.
例えば言語``ja''(日本語を想定)に関しては,\luavar{style.languages.ja}以下に以下のように設定します.
\begin{lstlisting}
-- 日本語用設定
style.languages.ja = {} -- テーブルを作成しておく
style.languages.ja.templates = {}
style.languages.ja.templates["article"]
    = "[<著者:|$<author>|>:<\\emph{|$<title>|}>:$<journal>:<|$<year>|年>]"
function style.languages.ja.formatters:author(c)
	if c.fields["author"] == nil then return nil end
	local a = Functions.split_names(c.fields["author"])
	if #a <= 2 then
		return Functions.make_name_list(a,"{vv~}{ff~}",{", ",", "},"他")
	else
		return Functions.make_name_list(a,"{vv~}{ff~}",{", ",", "},"他")
	end
end
-- 設定されていない項目(article以外のエントリータイプなど)に対しては
-- 言語によらない設定が適用される.

style.languages.ja.crossref = {}
-- クロスリファレンスの設定
style.languages.ja.label = {}
-- ラベルの設定
\end{lstlisting}

\subsection{言語決定}
言語決定は,langidフィールドにより決定されます.
langidフィールドがjaならば,言語jaの設定が使われます.
これを変更するには,\luavar{style.languages.<言語>.is}を定義します.
例えば,
\begin{lstlisting}
function style.languages.ja.is(c)
  if c.fields["language"] == "japanese" then 
    return true
  else
    return false
  end
end
\end{lstlisting}
とするとlanguagesフィールドがjapaneseのエントリーが日本語であると判定されます.
ただし,この場合languagesが空であるがlandidフィールドがjaであっても日本語とは判定されません.
これを避けるには上の条件分岐を
\begin{lstlisting}
c.fields["language"] == "japanese" or c.fields["langid"] == "ja"
\end{lstlisting}
とします.

\section{関数}\label{sec:関数}
有用そうな関数群やオリジナルの\BibTeX に存在していた関数が,モジュール\luavar{labtx-funcs}で定義されています.
\begin{lstlisting}
local Functions = require "labtx-funcs"
x = Functions.text_prefix(str,num)
\end{lstlisting}
のように使ってください.

\subsection{\luafunc{stable\_sort(list,comp)}}
配列\luavar{list}に対して,安定なソートを行います.
\luavar{comp}は比較関数です.
省略された場合は標準演算子 \luafunc{<} が使われます.

\subsection{\luafunc{text\_prefix(str,num)}}
\luavar{str}の先頭\luavar{num}バイトを返します.
ただし,文字を途中で切ることはなく,またコントロールシークエンス等や引数はバイト数に加算されません.
たとえば,
\begin{lstlisting}
text_prefix("aあい",2)
text_prefix("あいう",5)
\end{lstlisting}
はそれぞれ\luastring{aあ},\luastring{あい}を返します.\footnote{内部コードはUTF-8なので,\luastring*{あ}や\luastring*{い}は3byteです.この扱いはどうするか考え中…….}

\subsection{\luafunc{text\_length(str)}}
strのバイト数を返しますが,コントロールシークエンス等や引数は加算されません.

\subsection{\luafunc{string\_split(str,func)}}
検索関数\luavar{func}により\luavar{str}を分割して返します.
戻り値は二つの配列で,一つ目の配列には分割された文字列,二つ目の配列には分割文字列が入ります.
たとえば
\begin{lstlisting}
string_split("aXbYc",function(s) return s:find("[XY]") end)
\end{lstlisting}
は
\begin{lstlisting}
{"a","b","c"},{"X","Y"}
\end{lstlisting}
を返します.


\subsection{\luafunc{change\_case(str,format)}}
大文字小文字の変換を行います.ただし,中括弧の中は処理されません.
\luavar{format}は\luastring{t},\luastring{u},\luastring{l}のどれかで,
\begin{itemize}
\item \luastring{u},\luastring{l}はそれぞれ大文字,小文字への変換を表す.
\item \luastring{t}は小文字への変換を行うが,一文字目及び\luastring{: *}で表される文字の次の文字は変換されない.
\end{itemize}


\subsection{\luafunc{split\_names(names[,seps])}}
複数名の名前からなる文字列から,各人の名前の入った配列を得ます.
人と人との区切りを配列\luavar{seps}で与えることができます.(配列中のいずれかにマッチした部分で区切られる.)
\luavar{seps}のデフォルトは\verb|{"[aA][nN][dD]"}|です.

\subsection{\luafunc{get\_name\_parts(names)}}
名前からfirst name,last name,von part,jr partの四つの部分を抽出します.
戻り値は
\begin{lstlisting}
{first = <first part>, last = <last part>, von = <von part>, jr = <jr part>}
\end{lstlisting}
で,各々の部分は
\begin{lstlisting}
{parts = <array of name>, seps = <separator of names>}
\end{lstlisting}
です.%\verb|<array of name>|は各部分の名前が配列で,\verb|<separator of names>|はnames内で使われていた区切り記号を表します.
例えばvon-von Last Last, First, jrに対しては,次のように返ります.
\begin{lstlisting}
{
	first = {parts = {"First"}, seps = {}},
	last = {parts = {"Last","Last"}, seps = {" "}},
	von = {parts = {"von", "von"}, seps = {"-"}},
	jr = {pars = {"jr"}, seps = {}}
}
\end{lstlisting}

この関数は,次のルールに基づき名前を分解します.
\begin{enumerate}
\item \luastring{[ ,~\t%-]+}に該当するパターン\footnote{Luaの意味でのパターン}で区切り,配列を生成する.
\item 1で区切られた際に用いられた区切り文字のうち,最初の一文字がカンマ「,」のものの数を数える.この数に基づき,次の三つのパターンのどれかと見なす.
\begin{enumerate}
\item カンマがない:First von Lastのパターン.頭から見てvonと見なされるパターンの前までがFirst,後ろから見てvonと見なされるパターンの後ろまでがLast.vonがない場合は1で区切られたうちの最後の一つのみがLast.(ただし,区切り文字が\verb|"-"|のものはまとめて考える.例えば\luastring{First Last Last}のLastは\luastring{Last}であるが,\luastring{First Last-Last}ならば\luastring{Last-Last}である.)
\item カンマが一つ:von Last, Firstのパターン.von LastからLastを抜き出す処理は(a)と同じ.
\item カンマが二つ:von Last, Jr, Firstのパターン.von LastからLastを抜き出す処理は(a)と同じ.
\end{enumerate}
\item 2における「vonと見なされるパターン」とは,(基本的には)\footnote{実際には中括弧内や,コントロールシークエンスで定義されたアクセントなども考慮に入れる.}最初に現れたアルファベットが小文字であるもののことである.
\end{enumerate}


\subsection{\luafunc{forat\_name\_by\_parts(nameparts,format)}}
\luavar{format}にて指定された書式に基づき,名前の整形を行います.
\luavar{nameparts}は\luafunc{get\_name\_parts}で得られる戻り値と同じかたちで与えます.
\luavar{format}で与える書式は次の形です.
\begin{quote}
\begin{verbatim}
<str1>{<before1><name1><after1>}<str2>{<before2><name2><after2>}...
\end{verbatim}
\end{quote}
\begin{itemize}
\item \verb|<str1>|はそのまま出力される.
\item \verb|<name1>|は\luastring{l},\luastring{ll},\luastring{f},\luastring{ff},\luastring{v},\luastring{vv},\luastring{j},\luastring{jj}の何れか.Last name,First name,von part,jr partに対応し,二つ続いているものは名前全体を,そうでないものは短縮形を出力する.
\item \verb|<before1>|はそのまま出力される.ただし\verb|<name1>|に対応する部分がない場合,出力されない.
\item \verb|<after1>|は\verb|{<sep1>}<after1_>|か\verb|<after1_>|(中括弧なし)の何れかである.\verb|<sep1>|は\verb|<name1>|の各部分をつなぐ文字として使われ,\verb|<after1_>|は次の部分とのつなぐ文字として使われる.\verb|<sep1>|が省略された場合や,\luastring{~}であった場合は,空白(\luastring{ }か\luastring{~})が状況に応じて使われる.もし常に\luastring{~}を出力したい場合は,\luastring{~~}を指定する.
\item \verb|<str2>|等も同様.
\end{itemize}

\subsection{\luafunc{format\_name(name,format)}}
\BibTeX のformat.names\$と似た関数です.
中身は
\begin{lstlisting}
return forat_name_by_parts(get_name_parts(name),format)
\end{lstlisting}
です.

\subsection{\luafunc{make\_name\_list(namearray, format, separray[, etalstr])}}
複数人の名前の配列から文字列を作ります.
\luavar{separray}の長さを\luavar{k},\luavar{namearray}の長さを\luavar{n}とすると,
\begin{quote}
\begin{verbatim}
namearray[1]separray[1]namaearray[2]separray[2] .... 
namearray[n - k + 1]separray[2] ...
namearray[n - 1]separray[k]namearray[n]
\end{verbatim}
\end{quote}
という文字列を生成します.(実際には改行無し.)
ただし,\luavar{namearray}の各項は\luavar{format}に従い整形され(書式は\luafunc{format\_name\_by\_parts}と同様),またもし\luavar{namearray[n]}が\luastring{others}の場合は,\luavar{namearray[n]}は\luavar{etalstr}に置き換えられます.
デフォルトでは\luavar{etalstr}は空文字列です.

\subsection{\luafunc{remove\_TeX\_cs(s)}}
\luavar{s}から\TeX のコントロールシークエンスを取り除いた文字列を得ます.

\section{文献データ}\label{sec:文献データ}
文献データは以下のようなテーブルに格納されています.
変数名を\luavar{Citation}とします.
\begin{description}[style=nextline]
\item[\luavar{Citation.type}]
エントリータイプ
\item[\luavar{Citation.key}]
エントリーキー
\item[\luavar{Citation.fields}]
フィールドが格納されているテーブル.マクロなどが施された結果が返る.
\item[function \luafunc{Citation:clone()}]
自分の複製を作ります.
\item[function \luafunc{Citation:set\_field(key,cite,key1)}]
文献データ\luavar{cite}のキー\luastring{key1}のフィールドを\luastring{key}に設定します.
\item[function \luafunc{Citation:get\_raw\_field(key)}]
キー\luastring{key}のフィールドの生の値(マクロなど適用前)を返します.
\end{description}


\section{変数\luavar{BibTeX}}
変数\luavar{BibTeX}には,現在の\labtx の状態が格納されています.
\subsection{各種状態}
\begin{description}[style=nextline]
\item[\luavar{BibTeX.style\_name}]
スタイル名.
\item[\luavar{BibTeX.cites}]
引用されている文献一覧からなる配列.各々の中身は節\ref{sec:文献データ}の通り.
\item[\luavar{BibTeX.db}]
読み込まれたデータベースを表すテーブル.エントリーキー\luastring{key}には
\begin{lstlisting}
BibTeX.db["key"]
\end{lstlisting}
でアクセスできる.各々の中身は節\ref{sec:文献データ}の通り.
\item[\luavar{BibTeX.aux}]
\url{aux}ファイル名.
\item[\luavar{BibTeX.aux\_contents}]
\url{aux}ファイル名の中身.\url{aux}の各行の
\begin{verbatim}
\somecs{arg1}[arg2](arg3)
\end{verbatim}
という行から
\begin{lstlisting}
{somecs = {
	{arg = "arg1", open = "{", close = "}"},
	{arg = "arg2", open = "[", close = "]"},
	{arg = "arg3", open = "(", close = ")"}
}}
\end{lstlisting}
というテーブルが生成されて,ここに格納されている.
括弧は上記の\luastring{{}}, \luastring{[]}, \luastring{()}が認識され,対応がとれているものとして扱われる.
\end{description}

\subsection{関数}
\begin{description}[style=nextline]
\item[\luafunc{BibTeX:output(str)}]
\url{bbl}への出力を行う.
\item[\luafunc{BibTeX:outputline(str)}]
\url{bbl}への一行出力を行う.
\item[\luafunc{BibTeX:outputthebibliography(style)}]
\luavar{style}に従い文献データを出力する.
\item[\luafunc{BibTeX:warning(str)}]
文字列\luavar{str}を警告として出力する.出力は標準出力および\url{blg}に対して行われる.
\item[\luafunc{BibTeX:error(str,exit\_code)}]
文字列\luavar{str}をエラーとして出力し,終了コード\luavar{exit\_code}でプログラムを終了する.
出力は標準エラー出力および\url{blg}に対して行われる.
\item[\luafunc{BibTeX:log(str)}]
\url{blg}に\luavar{str}を出力する.
\item[\luafunc{BibTeX:message(str)}]
標準出力に\luavar{str}を出力する.

\end{description}

\section{デバッグ}
次のようにしておくと,デバッグに有用な情報がはき出されたりする……かもしれません.
\begin{lstlisting}
local labtxdebug = require "labtx-debug"
labtxdebug.debugmode = true -- デバッグモードON

-- 以下スタイルファイルの処理
\end{lstlisting}


\end{document}


